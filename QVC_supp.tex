%%%%%%%%%%%%%%%%%%%%%%%%%%%%%%%%%%%%%%%%%%%%%%%%%%%%%%%%%%%%%%%%%%%%%
%% This is a (brief) model paper using the achemso class
%% The document class accepts keyval options, which should include
%% the target journal and optionally the manuscript type.
%%%%%%%%%%%%%%%%%%%%%%%%%%%%%%%%%%%%%%%%%%%%%%%%%%%%%%%%%%%%%%%%%%%%%
\documentclass[journal=jpclcd,manuscript=article
%,layout=twocolumn
]{achemso}
%\setkeys{acs}{articletitle = true}

%%%%%%%%%%%%%%%%%%%%%%%%%%%%%%%%%%%%%%%%%%%%%%%%%%%%%%%%%%%%%%%%%%%%%
%% Place any additional packages needed here.  Only include packages
%% which are essential, to avoid problems later. Do NOT use any
%% packages which require e-TeX (for example etoolbox): the e-TeX
%% extensions are not currently available on the ACS conversion
%% servers.
%%%%%%%%%%%%%%%%%%%%%%%%%%%%%%%%%%%%%%%%%%%%%%%%%%%%%%%%%%%%%%%%%%%%%
\usepackage[version=3]{mhchem} % Formula subscripts using \ce{}
\usepackage[T1]{fontenc}       % Use modern font encodings
\usepackage{physics}

%%%%%%%%%%%%%%%%%%%%%%%%%%%%%%%%%%%%%%%%%%%%%%%%%%%%%%%%%%%%%%%%%%%%%
%% If issues arise when submitting your manuscript, you may want to
%% un-comment the next line.  This provides information on the
%% version of every file you have used.
%%%%%%%%%%%%%%%%%%%%%%%%%%%%%%%%%%%%%%%%%%%%%%%%%%%%%%%%%%%%%%%%%%%%%
%%\listfiles

%%%%%%%%%%%%%%%%%%%%%%%%%%%%%%%%%%%%%%%%%%%%%%%%%%%%%%%%%%%%%%%%%%%%%
%% Place any additional macros here.  Please use \newcommand* where
%% possible, and avoid layout-changing macros (which are not used
%% when typesetting).
%%%%%%%%%%%%%%%%%%%%%%%%%%%%%%%%%%%%%%%%%%%%%%%%%%%%%%%%%%%%%%%%%%%%%

\renewcommand\thefigure{S\arabic{figure}} 
\renewcommand\thetable{S\arabic{table}}

% math
\newcommand*{\pd}[1]{\dfrac{\partial}{\partial #1}}
\newcommand*{\TT}{^{\mathrm{T}}}

\usepackage{color}
\newcommand*{\todo}[1]{\textbf{\textcolor{red}{#1}}}
\newcommand*{\new}[1]{\textcolor{blue}{#1}}
%%%%%%%%%%%%%%%%%%%%%%%%%%%%%%%%%%%%%%%%%%%%%%%%%%%%%%%%%%%%%%%%%%%%%
%% Meta-data block
%% ---------------
%% Each author should be given as a separate \author command.
%%
%% Corresponding authors should have an e-mail given after the author
%% name as an \email command. Phone and fax numbers can be given
%% using \phone and \fax, respectively; this information is optional.
%%
%% The affiliation of authors is given after the authors; each
%% \affiliation command applies to all preceding authors not already
%% assigned an affiliation.
%%
%% The affiliation takes an option argument for the short name.  This
%% will typically be something like "University of Somewhere".
%%
%% The \altaffiliation macro should be used for new address, etc.
%% On the other hand, \alsoaffiliation is used on a per author basis
%% when authors are associated with multiple institutions.
%%%%%%%%%%%%%%%%%%%%%%%%%%%%%%%%%%%%%%%%%%%%%%%%%%%%%%%%%%%%%%%%%%%%%
\author{Felix Plasser}
\email{felix.plasser@univie.ac.at}
%\affiliation{Department of Chemistry, Loughborough University, LE11 3TU, United Kingdom.}
%\altaffiliation{Institute for Theoretical Chemistry, Faculty of Chemistry, University of Vienna, W\"ahringerstr. 17, 1090 Vienna, Austria}
\author{Sandra G\'omez}
\author{Sebastian Mai}
\author{Leticia Gonz\'alez}
\email{leticia.gonzalez@univie.ac.at}
%\affiliation[University of Vienna]
%{Institute for Theoretical Chemistry, Faculty of Chemistry, University of Vienna, W\"ahringerstr. 17, 1090 Vienna, Austria}

%%%%%%%%%%%%%%%%%%%%%%%%%%%%%%%%%%%%%%%%%%%%%%%%%%%%%%%%%%%%%%%%%%%%%
%% The document title should be given as usual. Some journals require
%% a running title from the author: this should be supplied as an
%% optional argument to \title.
%%%%%%%%%%%%%%%%%%%%%%%%%%%%%%%%%%%%%%%%%%%%%%%%%%%%%%%%%%%%%%%%%%%%%
\title{\textit{Electronic Supporting Information for}\\*[1em]Highly efficient surface hopping dynamics using a linear vibronic coupling model}

%%%%%%%%%%%%%%%%%%%%%%%%%%%%%%%%%%%%%%%%%%%%%%%%%%%%%%%%%%%%%%%%%%%%%
%% Some journals require a list of abbreviations or keywords to be
%% supplied. These should be set up here, and will be printed after
%% the title and author information, if needed.
%%%%%%%%%%%%%%%%%%%%%%%%%%%%%%%%%%%%%%%%%%%%%%%%%%%%%%%%%%%%%%%%%%%%%
%\abbreviations{IR,NMR,UV}
%\keywords{American Chemical Society, \LaTeX}

%%%%%%%%%%%%%%%%%%%%%%%%%%%%%%%%%%%%%%%%%%%%%%%%%%%%%%%%%%%%%%%%%%%%%
%% The manuscript does not need to include \maketitle, which is
%% executed automatically.
%%%%%%%%%%%%%%%%%%%%%%%%%%%%%%%%%%%%%%%%%%%%%%%%%%%%%%%%%%%%%%%%%%%%%
\begin{document}

%%%%%%%%%%%%%%%%%%%%%%%%%%%%%%%%%%%%%%%%%%%%%%%%%%%%%%%%%%%%%%%%%%%%%
%% Start the main part of the manuscript here.
%%%%%%%%%%%%%%%%%%%%%%%%%%%%%%%%%%%%%%%%%%%%%%%%%%%%%%%%%%%%%%%%%%%%%

\section{LVC Parameters}

 \begin{table*}[h!]
 \label{tab:LVC}
 \caption{LVC-Parameters (eV) for SO$_2$ computed at the MR-CIS(6,6)/ANO-RCC-VDZP level of theory.\textsuperscript{a}}
 \begin{tabular}{lrrr|rr}
 \hline
  & \multicolumn{1}{c}{$\epsilon_n$} & \multicolumn{1}{c}{$\kappa_{1a_1}^{(n)}$} & \multicolumn{1}{c}{$\kappa_{2a_1}^{(n)}$} & \multicolumn{2}{|c}{$\lambda_{b_2}^{(mn)}$} \\
 \hline
 &&&&& \\*[-0.9em]
 $^1A_1$ &     0 &      0 &      0 & $^1A_1/^1B_2$ & -0.498472\\
 $^1B_1$ & 4.462792 & 0.032558 & 0.331565 & $^1B_1/^1A_2$ & 0.199193 \\ 
$^1A_2$ & 4.847730 & -0.217356 & 0.504755 \\ 
$^1B_2$ & 6.805345 & -0.121284 & 0.482766 \\ 
$^3B_1$ & 3.649723 & 0.078288 & 0.222898 & $^3B_1/^3A_2$  & 0.155580  \\ 
$^3B_2$ & 4.478220 & -0.136967 & 0.499509 \\ 
$^3A_2$ & 4.627331 & -0.219168 & 0.506301 \\
 \hline
 \end{tabular}
 
 \vspace{0.5em}
 \textsuperscript{a} The frequencies are: $\omega_{1a_1}=518.75\mathrm{cm}^{-1}$, $\omega_{1a_2}=1165.17\mathrm{cm}^{-1}$, $\omega_{1a_1}=1405.24\mathrm{cm}^{-1}$.
 \end{table*}

 \begin{table*}
 \label{tab:LVCD}
 \caption{LVC-Parameters (eV) for SO$_2$ computed at the MR-CISD(12,9)/ANO-RCC-VTZP level of theory.\textsuperscript{a,b}}
 \begin{tabular}{lrrr|rr}
 \hline
  & \multicolumn{1}{c}{$\epsilon_n$} & \multicolumn{1}{c}{$\kappa_{1a_1}^{(n)}$} & \multicolumn{1}{c}{$\kappa_{2a_1}^{(n)}$} & \multicolumn{2}{|c}{$\lambda_{b_2}^{(mn)}$} \\
 \hline
 &&&&& \\*[-0.9em]
$^1A_1$ &  & -0.017641 & -0.024908 & $^1A_1/^1B_2$ & 0.451491 \\ 
$^1B_1$ & 4.226686 & 0.028991 & 0.294386 & $^1B_1/^1A_2$ & -0.151271 \\ 
$^1A_2$ & 4.595060 & -0.215731 & 0.444536 &  &  \\ 
$^1B_2$ & 8.403901 & -0.130622 & 0.463663 &  &  \\ 
$^3B_1$ & 3.349913 & 0.070993 & 0.189128 & $^3B_1/^3A_2$ & -0.099622 \\ 
$^3B_2$ & 4.210887 & -0.142804 & 0.465614 &  &  \\ 
$^3A_2$ & 4.356258 & -0.217377 & 0.440914 &  &  \\ 
 \hline
 \end{tabular}
 
 \vspace{0.5em}
 \textsuperscript{a} The frequencies are: $\omega_{1a_1}=518.75\mathrm{cm}^{-1}$, $\omega_{1a_2}=1165.17\mathrm{cm}^{-1}$, $\omega_{1a_1}=1405.24\mathrm{cm}^{-1}$.
 
 \textsuperscript{b} Reference geometry and vibrations determined at the MR-CIS level. Therefore, the $\kappa$ values for the ground state do not vanish.
 \end{table*}

%%%%%%%%%%%%%%%%%%%%%%%%%%%%%%%%%%%%%%%%%%%%%%%%%%%%%%%%%%%%%%%%%%%%%

 \begin{table*}[h!]
 \caption{Diabatic spin-orbit coupling parameters $\eta_{mn}$ (cm\textsuperscript{-1}) of SO$_2$ determined at the MR-CIS(6,6)/ANO-RCC-VDZP level of theory.}

\begin{tabular}{l||rrr|rrr|rrr}
\hline
 & $^3B_1(0)$ & $^3B_2(0)$ & $^3A_2(0)$ & $^3B_1(+)$ & $^3B_2(+)$ & $^3A_2(+)$ & $^3B_1(-)$ & $^3B_2(-)$ & $^3A_2(-)$ \\ 
\hline\hline
$^1A_1$ & 0.00 & 0.00 & -34.96 & 134.35 & 0.00 & 0.00 & 0.00 & 1.04 & 0.00 \\ 
$^1B_1$ & 0.00 & -23.83 & 0.00 & 0.00 & 0.00 & 0.00 & 0.00 & 0.00 & -52.02 \\ 
$^1A_2$ & 0.00 & 0.00 & 0.00 & 0.00 & -48.03 & 0.00 & 48.02 & 0.00 & 0.00 \\ 
$^1B_2$ & 6.69 & 0.00 & 0.01 & -0.01 & 0.00 & -65.74 & 0.00 & 0.00 & 0.00 \\ 
\hline\hline
$^3B_1(0)$ & 0.00 & 0.00 & 0.00 & 0.00 & 0.00 & -49.62 & 0.00 & 0.00 & 0.00 \\ 
$^3B_2(0)$ & 0.00 & 0.00 & 0.00 & 0.00 & 0.00 & 0.00 & 0.00 & 0.00 & -52.71 \\ 
$^3A_2(0)$ & 0.00 & 0.00 & 0.00 & 49.62 & 0.00 & 0.00 & 0.00 & 52.71 & 0.00 \\ 
\hline
$^3B_1(+)$ & 0.00 & 0.00 & 49.62 & 0.00 & 0.00 & 0.00 & 0.00 & -17.69 & 0.00 \\ 
$^3B_2(+)$ & 0.00 & 0.00 & 0.00 & 0.00 & 0.00 & 0.00 & 17.69 & 0.00 & 0.00 \\ 
$^3A_2(+)$ & -49.62 & 0.00 & 0.00 & 0.00 & 0.00 & 0.00 & 0.00 & 0.00 & 0.00 \\ 
\hline
$^3B_1(-)$ & 0.00 & 0.00 & 0.00 & 0.00 & 17.69 & 0.00 & 0.00 & 0.00 & 0.00 \\ 
$^3B_2(-)$ & 0.00 & 0.00 & 52.71 & -17.69 & 0.00 & 0.00 & 0.00 & 0.00 & 0.00 \\ 
$^3A_2(-)$ & 0.00 & -52.71 & 0.00 & 0.00 & 0.00 & 0.00 & 0.00 & 0.00 & 0.00 \\ 

\hline
 \end{tabular}
 \end{table*}

%%%%%%%%%%%%%%%%%%%%%%%%%%%%%%%%%%%%%%%%%%%%%%%%%%%%%%%%%%%%%%%%%%%%%
 \begin{table*}[h!]
 \caption{Diabatic spin-orbit coupling parameters $\eta_{mn}$ (cm\textsuperscript{-1}) of SO$_2$ determined at the MR-CISD(12,9)/ANO-RCC-VDZP level of theory.}
 
\begin{tabular}{l||rrr|rrr|rrr}

\hline
 & $^3B_1(0)$ & $^3B_2(0)$ & $^3A_2(0)$ & $^3B_1(+)$ & $^3B_2(+)$ & $^3A_2(+)$ & $^3B_1(-)$ & $^3B_2(-)$ & $^3A_2(-)$ \\ 
\hline\hline
$^1A_1$ & 0.00 & -0.03 & -45.87 & -146.25 & 0.00 & 0.03 & 0.00 & -0.45 & 0.01 \\ 
$^1B_1$ & 0.00 & 26.93 & -0.01 & 0.00 & -0.02 & 0.00 & -0.02 & -0.01 & -54.90 \\ 
$^1A_2$ & 0.00 & 0.01 & 0.00 & -0.01 & -44.05 & -0.03 & 51.47 & 0.00 & 0.09 \\ 
$^1B_2$ & 0.03 & -0.22 & -36.46 & -126.13 & 0.61 & 0.21 & 0.53 & -0.44 & 0.05 \\ 
\hline\hline
$^3B_1(0)$ & 0.00 & 0.00 & 0.00 & 0.00 & 0.02 & 52.85 & 0.00 & 0.00 & 0.00 \\ 
$^3B_2(0)$ & 0.00 & 0.00 & 0.00 & -0.02 & 0.00 & 0.00 & 0.00 & 0.00 & 48.10 \\ 
$^3A_2(0)$ & 0.00 & 0.00 & 0.00 & -52.85 & 0.00 & 0.00 & 0.00 & -48.10 & 0.00 \\ 
\hline
$^3B_1(+)$ & 0.00 & -0.02 & -52.85 & 0.00 & 0.00 & 0.00 & 0.00 & -20.60 & 0.00 \\ 
$^3B_2(+)$ & 0.02 & 0.00 & 0.00 & 0.00 & 0.00 & 0.00 & 20.60 & 0.00 & 0.05 \\ 
$^3A_2(+)$ & 52.85 & 0.00 & 0.00 & 0.00 & 0.00 & 0.00 & 0.00 & -0.05 & 0.00 \\ 
\hline
$^3B_1(-)$ & 0.00 & 0.00 & 0.00 & 0.00 & 20.60 & 0.00 & 0.00 & 0.00 & 0.00 \\ 
$^3B_2(-)$ & 0.00 & 0.00 & -48.10 & -20.60 & 0.00 & -0.05 & 0.00 & 0.00 & 0.00 \\ 
$^3A_2(-)$ & 0.00 & 48.10 & 0.00 & 0.00 & 0.05 & 0.00 & 0.00 & 0.00 & 0.00 \\ 
\hline
\end{tabular}
\end{table*}

%%%%%%%%%%%%%%%%%%%%%%%%%%%%%%%%%%%%%%%%%%%%%%%%%%%%%%%%%%%%%%%%%%%%%
%% The appropriate \bibliography command should be placed here.
%% Notice that the class file automatically sets \bibliographystyle
%% and also names the section correctly.
%%%%%%%%%%%%%%%%%%%%%%%%%%%%%%%%%%%%%%%%%%%%%%%%%%%%%%%%%%%%%%%%%%%%%
%\bibliography{allrefs}


\end{document}
